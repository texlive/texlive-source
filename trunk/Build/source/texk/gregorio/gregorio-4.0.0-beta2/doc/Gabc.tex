\section{The GABC File}

gabc is a simple notation based exclusively on ASCII characters that
enables the user to describe Gregorian chant scores. The name
\textit{gabc} was given in reference to the
\href{http://www.walshaw.plus.com/abc/}{ABC} notation for modern
music.

The gabc notation was developed by a monk of the
\href{http://www.barroux.org}{Abbey of Sainte Madeleine du Barroux}
and has been improved by Élie Roux and by other monks of the same
abbey to produce the best possible notation.

This section will cover the elements of a gabc file.

\subsection{File Structure}
Files written in gabc have the extension \texttt{.gabc} and have the
following structure:

\begin{lstlisting}
name: incipit;
gabc-copyright: copyright on this gabc file;
score-copyright: copyright on the source score;
office-part: introitus/...;
occasion: in church calendar;
meter: for metrical hymns;
commentary: source of words;
arranger: name of arranger;
gabc-version: 0.9.3;
author: if known;
date: xi c;
manuscript: ms name;
manuscript-reference: e.g. CAO reference;
manuscript-storage-place: library/monastery;
book: from which score taken;
transcriber: writer of gabc;
transcription-date: 2009;
initial-style: 1;
user-notes: whatever other comments you wish to make;
mode: 6;
annotation: IN.;
annotation: 6;
%%
(clef) text(notes)
\end{lstlisting}

In each case, replace whatever is between the colon and semi-colon
\texttt{(:...;)} with the appropriate character string. Of these attributes,
only name is mandatory. Descriptions of how these header fields are
intended to be used are below. If you wish to write a value over
several lines, omit the semicolon at the end of the first line, and
end the attribute value with \texttt{;;} (two semicolons).

\subsection{Header}

Here is a detailed description of each header field:

\begin{description}
\item[name] This is the name of the piece, in almost all cases the
	incipit, the first few words. In the case of the mass ordinary, the
	form as \texttt{Kyrie X Alme Pater} or \texttt{Sanctus XI} is
	recommended where appropriate. \textbf{This field is required.}
\item[mode] The mode of the piece. This should normally be an arabic
	number between 1 and 8, but may be any text required for unusual
	cases. The mode number will be converted to roman numerals and
	placed above the initial unless one of the following conditions are
	met:
	\begin{itemize}
	\item There is a \verb=\greannotation= defined immediatly prior to \verb=\gregorioscore=.
	\item The \texttt{annotation} header field is defined.
	\end{itemize}
\item[annotation] The annotation is the text to appear above the
	initial letter. Usually this is an abbreviation of the office-part
	in the upper line, and an indication of the mode (and differentia
	for antiphons) in the lower. Either one or two annotation fields may
	be used; if two are used, the first is the upper line, the second
	the lower. Example:
	\begin{lstlisting}
		annotation:Ad Magnif.;
		annotation:VIII G;
	\end{lstlisting}
	Full \TeX markup is accepted:
	\begin{lstlisting}
		annotation:{\color{red}Ad Magnif.};
		annotation:{\color{red}VIII G};
	\end{lstlisting}
	If the user already defined annotation(s) in the main TeX file via
	\verb=\greannotation= then the \texttt{annotation} header field will not
	overwrite that definition.

\end{description}

\subsection{Lyric Centering}

Gregorio centers the text of each syllable around the first note of each
syllable.  There are two basic modes: \verb:\setlyriccentering{syllable}:
and \verb:\setlyriccentering{vowel}:.  In \texttt{syllable} mode, the
entire syllable is centered around the first note.  This is common in
modern music.  In \texttt{vowel} mode, the vowel sound of the syllable is
centered around the first note.  This is common in most Gregorian chant
books with text in Latin.\bigskip

The default rules built into Gregorio for \texttt{vowel} mode are for
Ecclesiatical Latin and work fairly well (though not perfectly) for
other languages (especially Romance languages).  However, Gregorio
provides a gabc \texttt{language} header which allows the language of
the lyrics to be set.  Gregorio will look for a file named
\texttt{gregorio-vowels.dat} in your working directory or amongst the
GregorioTeX files.  If it finds the requested language (matched in a
\emph{case-sensitive} fashion) in one of these files (henceforth called
vowel files), Gregorio will use the rules contained within for vowel
centering.  If it cannot find the requested language in any of the vowel
files or is unable to parse the rules, Gregorio will fall back on the
Latin rules.  If multiple vowel files have the desired language,
Gregorio will use the first matching language section in the first
matching file, according to Kpathsea order.  You may wish to enable
verbose output (by passing the \texttt{-v} argument to
\texttt{gregorio}), if there is a problem, for more information.\bigskip

The vowel file is a list of statements, each starting with a keyword and
ending with a semicolon (\texttt{;}).  Multiple statements with the same
keyword are allowed, and all will apply.  Comments start with a hash
symbol (\texttt{\#}) and end at the end of the line.\bigskip

In general, Gregorio does no case folding, so the keywords and language
names are case-sensitive and both upper- and lower-case characters
should be listed after the keywords if they should both be considered in
their given categories.\bigskip

The keywords are:

\begin{description}

\item[alias]

The \texttt{alias} keyword indicates that a given name is an alias for a
given language.  The \texttt{alias} keyword must be followed by the name
of the alias (enclosed in square brackets), the \texttt{to} keyword, the
name of the target language (enclosed in square brackets), and a
semicolon.  Since gregorio reads the vowel files sequentially, aliases
should precede the language they are aliasing, for best performance.

\item[language]

The \texttt{language} keyword indicates that the rules which follow are
for the specified language.  It must be followed by the language name,
enclosed in square brackets, and a semicolon.  The language specified
applies until the next language statement.

\item[vowel]

The \texttt{vowel} keyword indicates that the characters which follow,
until the next semicolon, should be considered vowels.

\item[prefix]

The \texttt{prefix} keyword lists strings of characters which end in a
vowel, but when followed by a sequence of vowels, \emph{should not} be
considered part of the vowel sound.  These strings follow the keyword
and must be separated by space and end with a semicolon.  Examples of
prefixes include \emph{i} and \emph{u} in Latin and \emph{qu} in
English.

\item[suffix]

The \texttt{suffix} keyword lists strings of characters which don't
start with a vowel, but when appearing after a sequence of vowels,
\emph{should} be considered part of the vowel sound.  These strings
follow the keyword and must be separated by space and end with a
semicolon.  Examples of suffixes include \emph{w} and \emph{we} in
English and \emph{y} in Spanish.

\item[secondary]

The "secondary" keyword lists strings of characters which do not contain
vowels, but for which, when there are no vowels present in a syllable,
define the center of the syllable.  These strings follow the keyword and
must be separated by space and end with a semicolon.  Examples of
secondary sequences include \emph{w} from Welsh loanwords in English and
the syllabic consonants \emph{l} and \emph{r} in Czech.

\end{description}

By way of example, here is a vowel file that works for English:

\begin{lstlisting}
alias [english] to [English];

language [English];

vowel aàáAÀÁ;
vowel eèéëEÈÉË;
vowel iìíIÌÍ;
vowel oòóOÒÓ;
vowel uùúUÙÚ;
vowel yỳýYỲÝ;
vowel æǽÆǼ;
vowel œŒ;

prefix qu Qu qU QU;
prefix y Y;

suffix w W;
suffix we We wE WE;

secondary w W;
\end{lstlisting}
